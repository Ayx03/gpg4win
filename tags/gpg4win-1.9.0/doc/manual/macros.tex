\newcommand{\Button}[1]{[\,#1\,]}
\newcommand{\Menu}[1]{\emph{#1}}
\newcommand{\Filename}[1]{\texttt{#1}}

% Some people still insist that Email should be written as two words
% for whatever reasons.  Maybe is is that they don't like the game of
% Teekesselchen and fear the ambiguity between email and enamel (which
% uses the same word in German).  

%\newcommand{\Email}{E-Post}
\newcommand{\Email}{E-Mail}
%\newcommand{\Email}{EMail}
%\newcommand{\Email}{Email}

\newcommand{\EchelonUrl}{http://www.heise.de/tp/r4/artikel/6/6928/1.html}
\newcommand{\EinsteigerPDFURL}{http://wald.intevation.org/frs/download.php/134/gpg4win-fuer-einsteiger-2.0.0.RC1.pdf}
\newcommand{\DurchblickerPDFURL}{http://wald.intevation.org/frs/download.php/134/gpg4win-fuer-durchblicker-2.0.0.RC1.pdf}

\newcommand{\OtherBook}{\texorhtml{$\spadesuit$}{\xml{p}==\xmlent{gt}}}

\newcommand{\IncludeImage}[2][]{\texorhtml{%
\includegraphics[#1]{#2}%
}{%
\htmlimg{#2.png}%
}}

\newcommand{\HlxIcons}{}

\W\htmlpanelgerman
\W\extrasgerman
\W\dategerman
\W\captionsgerman
\htmlpanelfield{Inhalt}{hlxcontents}
\W\renewenvironment{thefootnotes}{\chapter{Fu�noten}
\W  \begin{description}}{\end{description}}


% Get the version information from another file.  That file is created
% by the configure script.
\input{version.tex}


