% was-istgpg4win.tex  - Einleitung beider Hadnb�cher.

\textbf{
Das Projekt Gpg4win (GNU Privacy Guard for Windows) ist eine vom
Bundesamt f�r Sicherheit in der Informationstechnik beauftragte
Email-Verschl�sselungssoftware. Gpg4win bezeichnet ein Gesamtpaket,
welches die folgenden Programme umfasst:
}

\textbf{GnuPG}: das Kernst�ck, die Verschl�sselungs-Software \newline
\textbf{GPA}: der GNU Privacy Assistent,
eine Schl�sselverwaltung \newline
\textbf{WinPT}: Schl�sselverwaltung, unterst�tzt auch Verschl�sselung
per Clipboard \newline
\textbf{GPGol}: ein Plugin f�r Microsoft Outlook, es integriert dort die
Bedienung von GnuPG \newline
\textbf{GPGee}: ein Plugin f�r den Windows Explorer, per rechter Maustaste
k�nnen Dateien verschl�sselt werden \newline
\textbf{Sylpheed-Claws}: ein komplettes Email-Programm mit integrierter
GnuPG-Bedienung \newline

Mit dem Verschl�sselungsprogramm GnuPG (GNU Privacy Guard) kann
jedermann Emails sicher, einfach und kostenlos verschl�sseln. GnuPG
kann ohne jede Restriktion privat oder kommerziell benutzt werden. Die
von GnuPG eingesetzte Verschl�sselungstechnologie ist sehr
sicher und kann nach dem heutigen
Stand von Forschung und Technik nicht gebrochen werden.

GnuPG ist Freie Software\footnote{oft ungenau auch als Open Source
  Software bezeichnet}. Das bedeutet, dass jedermann das Recht hat, sie
nach Belieben kommerziell oder privat zu nutzen.  Jedermann darf den
Quellcode, also die eigentliche Programmierung des Programms, genau
untersuchen und auch selbst �nderungen durchf�hren und diese
weitergeben.\footnote{Obwohl dies ausdr�cklich erlaubt ist, sollte man
  ohne ausreichendes Fachwissen nicht leichtfertig �nderungen
  durchf�hren da hierdurch die Sicherheit der Software beeintr�chtigt
  werden kann.}

F�r eine Sicherheits-Software ist diese garantierte Transparenz des
Quellcodes eine unverzichtbare Grundlage. Nur so l��t sich die
Vertrauensw�rdigkeit eines Programmes pr�fen.

GnuPG basiert auf dem internationalen Standard OpenPGP (RFC 2440), ist
vollst�ndig kompatibel zu PGP und benutzt die gleiche Infrastruktur
(Schl�sselserver etc.).

PGP ("`Pretty Good Privacy"') ist keine Freie Software, sie war
lediglich vor vielen Jahren kurzzeitig zu �hnlichen Bedingungen wie
GnuPG erh�ltlich.  Diese Version entspricht aber schon lange nicht
mehr dem Stand der Technik.

Weitere Informationen zu GnuPG und den Projekten der Bundesregierung
zum Schutz des Internets finden Sie auf der Website
\W\xlink{www.bsi-fuer-buerger.de}{http://www.bsi-fuer-buerger.de}
\T\href{http://www.bsi-fuer-buerger.de}{www.bsi-fuer-buerger.de}
des
Bundesamtes f�r Sicherheit in der Informationstechnik.

