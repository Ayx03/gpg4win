% what-is-gpg4win.tex  - Introduction used by both manuals


\textbf{
Gpg4win (GNU Privacy Guard for Windows) is an email encryption software.
It is the result of a project initiated by the Federal Office for
Information Security, and includes the following components: 
}

\textbf{GnuPG}: its key component, the encryption software \newline
\textbf{GPA}: GNU Privacy Assistant,
a key manager \newline
\textbf{WinPT}: Key Manager, which also supports encryption via your
Clipboard \newline
\textbf{GPGol}: a plugin for Microsoft Outlook which integrates the
operation of GnuPG \newline
\textbf{GPGee}: a plugin for Windows Explorer which allows encryption of
data by right-clicking on your mouse \newline
\textbf{Claws Mail}: a complete email program with integrated GnuPG
operation \newline

The encryption program GnuPG (GNU Privacy Guard) provides you with a
secure, simple and free method of email encryption. It can be used
privately or commercially without restrictions.  The encryption
technology used by GnuPG is extremely secure and cannot be broken using
current technology.

GnuPG is a free software\footnote{sometimes incorrectly identified as
Open Source Software}. This means that anyone can use the software for
private or commercial purposes, as well as analyze or change the source
codes (ie. the actual programming commands), and distribute the
same.\footnote{However, keep in mind that a fair amount of technical
knowledge is required to change the program, otherwise the program's
security may be compromised.}

The transparency of the source code forms an essential part of a
security software, as it is the only way to verify the trustworthiness
of the program.


GnuPG is based on the international standard OpenPGP (RFC 2440), is
fully compatable with PGP and uses the same infrastructure (key server
etc.).

PGP ("`Pretty Good Privacy"') is not free software; many years ago it
was available on a temporary basis under similar conditions as GnuPG,
but is no longer considered state-of-the-art.  

Additional information regarding GnuPG and other projects undertaken by
the german government in the area of Internet security can be found on
the website of the Federal Office for Information Security \W
\xlink{www.bsi-fuer-buerger.de}{http://www.bsi-fuer-buerger.de}
\T\href{http://www.bsi-fuer-buerger.de}{www.bsi-fuer-buerger.de}



